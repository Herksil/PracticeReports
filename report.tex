\documentclass[12pt,a4paper]{scrartcl} 
\usepackage[utf8]{inputenc}
\usepackage[english,russian]{babel}
\usepackage[colorlinks,urlcolor=blue]{hyperref}
\begin{document}
\thispagestyle{empty}
\begin{center}
Министерство образования и науки Украины
\end{center}
\begin{center}
Одесский национальный университет им. И.И.Мечникова
\end{center}
\par\bigskip
\par\bigskip
\par\bigskip
\par\bigskip
\par\bigskip
\par\bigskip
\par\bigskip
\par\bigskip
\par\bigskip
\par\bigskip
\par\bigskip
\par\bigskip
\par\bigskip
\par\bigskip
\par\bigskip
\begin{center}
\Large\textbf{Отчет по учебной практике}
\end{center}
\par\bigskip
\par\bigskip
\par\bigskip
\par\bigskip
\par\bigskip
\par\bigskip
\par\bigskip
\par\bigskip
\begin{flushright}
Выполнил:\\
Студент III курса\\
Специальности «Прикладная математика»\\
Гайдей Р.В.\\
Преподаватель:\\
Огуленко А.П.
\end{flushright}
\par\bigskip
\par\bigskip
\par\bigskip
\par\bigskip
\par\bigskip
\par\bigskip
\par\bigskip
\par\bigskip
\par\bigskip
\par\bigskip
\begin{center}
г. Одесса, 2016
\end{center}
\newpage
\begin{center}
\Large\textbf{Постановка задачи}
\end{center}
\begin{enumerate}
\item Знакомство с работой Git
\item Знакомство с Python
\item Освоение LaTeX
\end{enumerate}
\newpage
\begin{center}
\Large\textbf{Ход работы}
\end{center}
\parbox{14,5 cm}{\parindent=1 cm В период учебной летней практики 27.06.16-10.07.16 мною было освоено следующее:}
\begin{enumerate}
\item Git
\item Python
\item LaTeX
\item Beamer
\end{enumerate}
\parbox{14,5 cm}{\parindent=1 cm Данная учебная практика была довольно полезной для меня, так как работа с Git – это основа при командной разработке любого приложения, язык программирования Python в последнее время набирает все большую популярность, а с помощью LaTeX и Beamer любой математик должен уметь верстать свои научные труды и презентации. }
\end{document}